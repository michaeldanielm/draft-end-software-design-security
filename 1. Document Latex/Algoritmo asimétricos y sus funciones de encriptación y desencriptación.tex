\documentclass[a4paper]{article}
% Esto es para que el LaTeX sepa que el texto está en español:
\usepackage[spanish]{babel}
\selectlanguage{spanish}
\usepackage[utf8]{inputenc}
\usepackage[T1]{fontenc}



%% Asigna un tamaño a la hoja y los márgenes
\usepackage[a4paper,top=3cm,bottom=2cm,left=3cm,right=3cm,marginparwidth=1.75cm]{geometry}

%% Paquetes de la AMS
\usepackage{amsmath, amsthm, amsfonts}

\usepackage{graphicx}
\usepackage[colorinlistoftodos]{todonotes}
\usepackage[colorlinks=true, allcolors=blue]{hyperref}

%% Primero escribimos el título
\title{Algoritmo asimétricos y sus funciones de encriptación y desencriptación}
\author{Michael Daniel Murillo López\\
  \small Corporación Universitaria Minuto de Dios -UNIMINUTO\\
  \small mmurillolo1@unimiuto.edu.co\\
  \small Bogotá DC - Colombia
  \date{}
}


\begin{document}
\maketitle
 
\begin{abstract}
Los algoritmos  asimétricos son técnicas inteligentes que se usan para la protección  de datos  y  son parte fundamental al momento de definir una seguridad de un sistema. Su uso ha permitido un gran niveles de seguridad donde es necesario garantizando privacidad e incluso anonimato. En los algoritmos asimétricos se usa un par de claves para el envío de mensajes. Las dos claves pertenecen a la misma persona que ha enviado el mensaje. Una clave es pública y se puede entregar a cualquier persona, la otra clave es privada y el propietario debe guardarla de modo que nadie tenga acceso a ella. Además, los métodos criptográficos garantizan que esa pareja de claves sólo se puede generar una vez, de modo que se puede asumir que no es posible que dos personas hayan obtenido casualmente la misma pareja de claves.

\end{abstract}

%% Iniciamos "secciones" que servirán como subtítulos
%% Nota que hay otra manera de añadir acentos
\section{Introducci\'on}

Es un algoritmo que modifica los datos de un documento con el objeto de alcanzar algunas características de seguridad como autenticación, integridad y confidencialidad.

\section{Que son las llaves publicas y privadas en los algoritmos asimétricos}

Primero tienes que cargar el archivo de imagen desde su computadora usando el enlace de carga del menú del proyecto. Luego usando el comando 'includegraphics' podrás incluirlo en el documento. Con el entorno de figura y el comando de título podrás agregar un número y un título a la figura. Mira el código de la Figura \ref{fig:tesla} en esta sección para ver un ejemplo.

\begin{figure}
\centering
\caption{\label{fig:tesla}Esta imagen se añadió en el menú Project.}
\end{figure}

\section{Tipos de Algoritmos Asimétricos}

\subsection{Diffie-Hellman}

Primero tienes que cargar el archivo de imagen desde su computadora usando el enlace de carga del menú del proyecto. Luego usando el comando 'includegraphics' podrás incluirlo en el documento. Con el entorno de figura y el comando de título podrás agregar un número y un título a la figura. Mira el código de la Figura \ref{fig:tesla} en esta sección para ver un ejemplo.

\begin{figure}
\centering
\caption{\label{fig:tesla}Esta imagen se añadió en el menú Project.}
\end{figure}


%\subsection{¿Cómo añadir comentarios?}

Puedes añadir comentarios en el ícono + del menú de arriba.

Para responder a un comentario, simplemente da click en Reply en Rich Text.

También pueden añadirse comentarios en el margen del pdf compilado con el comando todo \todo{¡Comment en el margen!}, como se muestra en el ejemplo de la derecha. También puedes añadirlos dentro del texto:

\todo[inline, color=green!40]{Este es un comentario dentro del texto.}

\subsection{RSA}

Usa los comandos table y tabular para iniciar una tabla simple --- mira la tabla~\ref{tab:tabla ejemplo}, como ejemplo. 

\begin{table}
\centering
\begin{tabular}{l c r} 
%nùmero de columnas: 3
l para left & c para centro & r para derecha \\ \hline
Ejemplo & Centrado & Alineado a la\\
Izquierda & 13 & Derecha
\end{tabular}
\caption{\label{tab:tabla ejemplo}Una simple tabla.}
\end{table}
\subsection{DSA}

Usa los comandos table y tabular para iniciar una tabla simple --- mira la tabla~\ref{tab:tabla ejemplo}, como ejemplo. 

\begin{table}
\centering
\begin{tabular}{l c r} 
%nùmero de columnas: 3
l para left & c para centro & r para derecha \\ \hline
Ejemplo & Centrado & Alineado a la\\
Izquierda & 13 & Derecha
\end{tabular}
\caption{\label{tab:tabla ejemplo}Una simple tabla.}
\end{table}

\subsection{Cifrado ElGamal}

Usa los comandos table y tabular para iniciar una tabla simple --- mira la tabla~\ref{tab:tabla ejemplo}, como ejemplo. 

\begin{table}
\centering
\begin{tabular}{l c r} 
%nùmero de columnas: 3
l para left & c para centro & r para derecha \\ \hline
Ejemplo & Centrado & Alineado a la\\
Izquierda & 13 & Derecha
\end{tabular}
\caption{\label{tab:tabla ejemplo}Una simple tabla.}
\end{table}

\subsection{Cifrado Criptografía de curva elíptica}

Usa los comandos table y tabular para iniciar una tabla simple --- mira la tabla~\ref{tab:tabla ejemplo}, como ejemplo. 

\begin{table}
\centering
\begin{tabular}{l c r} 
%nùmero de columnas: 3
l para left & c para centro & r para derecha \\ \hline
Ejemplo & Centrado & Alineado a la\\
Izquierda & 13 & Derecha
\end{tabular}
\caption{\label{tab:tabla ejemplo}Una simple tabla.}
\end{table}

\subsection{Criptosistema de Merkle-Hellman}

\LaTeX{} es buenísimo para escribir ecuaciones. Para escribir variables o ecuaciones dentro del texto lo podemos poner entre signos de pesos y luego podemos seguir escribiendo, esto funciona si queremos escribir un símbolo como $\nabla$, $\pi$, $\beta$, $\Omega$, $\aleph$, etc.
\begin{equation}
\sum_{n=0}^\infty \frac{x^n}{n!}=e^x
\end{equation}
\begin{equation}
\int_{0}^{1}dx=1
\end{equation}
\begin{equation}
e^{i\pi}+1=0
\end{equation}
Si queremos citar al gran Maxwell, lo podemos hacer como en la ecuación \ref{eq:Maxwell}:
\begin{equation}
\nabla\times\mathbf{E}+\frac{\partial\mathbf{B}}{\partial t}=0\label{eq:Maxwell}
\end{equation}

A continuación se añade un ejemplo de un desarrollo:
Con este preámbulo llevamos a cabo la siguiente transformación de los operadores $\hat{a}_{\ell}$

\begin{equation}
\hat{b}_{m}^{\dagger}=\sum_{\ell}U_{m}^{\ell}\hat{a}_{\ell}^{\dagger}
\end{equation}

donde $U_{m}^{\ell}$ es un elemento de la matriz unitaria $\mathbf{U}$.

Calculamos ahora su hermitiano conjugado
\begin{align}
\hat{b}_{m} & =\left(\sum_{\ell}U_{m}^{\ell}\hat{a}_{\ell}^{\dagger}\right)^{\dagger}\label{eq:bm}\\
 & =\sum_{\ell}\left(U_{m}^{\ell}\hat{a}_{\ell}^{\dagger}\right)^{\dagger}\nonumber \\
 & =\sum_{\ell}\left(U_{m}^{\ell}\right)^{*}\hat{a}_{\ell}\nonumber \\
 & =\sum_{\ell}\left(U^{-1}\right)_{\ell}^{m}\hat{a}_{\ell},\label{eq:bSubM}
\end{align}

Ahora, para añadir una matriz:

$$
\begin{matrix} 
a & b \\
c & d 
\end{matrix}
\quad
\begin{pmatrix} 
a & b \\
c & d 
\end{pmatrix}
\quad
\begin{bmatrix} 
a & b \\
c & d 
\end{bmatrix}
\quad
\begin{vmatrix} 
a & b \\
c & d 
\end{vmatrix}
\quad
\begin{Vmatrix} 
a & b \\
c & d 
\end{Vmatrix}
$$
%% Por ejemplo, el triple producto escalar:
\begin{equation}
\vec{A}\cdot(\vec{B}\times\vec{C})=\begin{vmatrix}
A_x&A_y&A_z\\
B_x&B_y&B_z\\
C_x&C_y&C_z\\
\end{vmatrix}
\end{equation}

\subsection{Goldwasser-Micali}

Puedes añadir listas con numeración automática \dots

\begin{enumerate}
\item Como esta,
\item y como esta.
\end{enumerate}
\dots o con puntitos \dots
\begin{itemize}
\item Como este,
\item y como este.
\end{itemize}
\subsection{Cifrado Goldwasser-Micali-Rivest}

Usa los comandos table y tabular para iniciar una tabla simple --- mira la tabla~\ref{tab:tabla ejemplo}, como ejemplo. 

\begin{table}
\centering
\begin{tabular}{l c r} 
%nùmero de columnas: 3
l para left & c para centro & r para derecha \\ \hline
Ejemplo & Centrado & Alineado a la\\
Izquierda & 13 & Derecha
\end{tabular}
\caption{\label{tab:tabla ejemplo}Una simple tabla.}
\end{table}

\subsection{Cifrado extremo a extremo}

Usa los comandos table y tabular para iniciar una tabla simple --- mira la tabla~\ref{tab:tabla ejemplo}, como ejemplo. 

\begin{table}
\centering
\begin{tabular}{l c r} 
%nùmero de columnas: 3
l para left & c para centro & r para derecha \\ \hline
Ejemplo & Centrado & Alineado a la\\
Izquierda & 13 & Derecha
\end{tabular}
\caption{\label{tab:tabla ejemplo}Una simple tabla.}
\end{table}


\subsection{¿Cómo añado una lista de Citas y Referencias?}

Puedes subir un archivo \verb|.bib| que contenga todas tus referencias en estilo BibTeX (puedes buscar la bibliografía de un libro en google añadiendo 'bibtex' al final), creado con JabRef. Luego podrás hacer citas así: \cite{Griffiths:1492149}.

\bibliographystyle{abbrv}
\bibliography{sample}

\end{document}
